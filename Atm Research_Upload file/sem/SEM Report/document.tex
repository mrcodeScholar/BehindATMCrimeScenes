\documentclass[10pt, conference, compsocconf]{IEEEtran}
\usepackage{graphicx}
\usepackage{float}
\usepackage{amsmath}
\title{Structure equation modeling}

\begin{document}
	
	
\maketitle


\begin{figure}[h!]
	\centering
	\includegraphics[width=\linewidth]{semoriginal.png}
	\caption{SEM Model}
	\label{fig:Model}
\end{figure}

	
\section{Model Explanation}	
	
	
	\begin{enumerate}
		\item Python library used: Semopy
		\item  Objective function used: (GLS) Generalized least square
		\item Optimization method: (SLSQP) Sequential Least Squares Quadratic Programming
		\item Objective value: 0.933 the objective function is typically a measure of fit between the model and the observed data.
		\item Number of iterations: 220
	\end{enumerate}
\textbf{Equation used}
\begin{subequations}
	\begin{align}
		D &\sim \text{com\_dens} + \text{age} + \text{income} \nonumber \\
		&\qquad + \text{unp\_rate} + \text{density} \\
		G &\sim \text{police\_dis} + \text{mway\_dist} \\
		\text{success\_attack} &\sim G + D
	\end{align}
\end{subequations}
	
\textbf{Bias correction}	
	The function \textbf{sem.bias\_correction(model, n=50, resample\_mean=True)} applies bias correction to a fitted structural equation model (\texttt{model}). Bias correction is a statistical technique used to adjust the model parameters to account for any bias introduced during the estimation process.\\
	
	
	
	\textbf{P\_value:} indicates how statistically significant the relation between two variable are. p\_value must be less than 0.5 or 0.1 for the relation between two variables to be considered statistically significant.\\
	The stand alone values written beside each arrow signifies the magnitude and direction of strength.
	 
	 
	 For some variables the p\_value is not calculated upon investigation it was identified that this can happen when the dataset is limited and model complexity is high.\\
	 \newpage
	 In Structural Equation Modeling (SEM), an arrow pointing from one variable to another represents a hypothesized causal relationship. In our case, the arrow pointing from "G" to "success\_attack" suggests that there is a hypothesized causal relationship from the latent variable "G" to the observed variable "success\_attack."
	 
	 The p-value associated with this arrow (0.00) likely indicates the statistical significance of this relationship. A p-value of 0.00 suggests that the relationship is statistically significant at conventional significance levels (such as Alpha = 0.05), meaning that it's highly unlikely to have occurred by chance.\\
	 The dotted bi-directional arrow from "G" to "D" in the SEM model indicates a hypothesized relationship between these two latent variables. The fact that the arrow is bi-directional suggests that there is a hypothesized bidirectional causal relationship between "G" and "D," meaning that they influence each other.\\
	 A p-value of 0.12 suggests that the relationship between "G" and "D" is not statistically significant at conventional significance levels (such as Alpha = 0.05). This means that there is insufficient evidence to conclude that the relationship between "G" and "D" is different from zero, based on the current data and model.\\
	 For "G" and "D" in this case, a positive value of 13.694 suggests a positive relationship between "G" and "D," indicating that changes in one variable are associated with changes in the other variable.\\
	 the demographic variable is pointing to the observed variable "success\_attack" with a value of -21.950. This suggests that there is a hypothesized causal relationship from the demographic variable to the outcome variable "success\_attack."
	The negative value (-21.950) indicates the direction and strength of this relationship. In this case, a negative value suggests that increases in the demographic variable are associated with decreases in the outcome variable "success\_attack." The magnitude of the value (-21.950) indicates the strength of this relationship. The p-value associated with this relationship (0.00) indicates the statistical significance of the estimated parameter.\\
	The SEM analysis reveals that demographic variables significantly impact success\_attack, with higher demographics linked to lower success\_attack rates. Additionally, a robust positive relationship exists between "G" and success\_attack, indicating that higher values of "G" correlate with increased success\_attack rates. However, the bidirectional relationship between "G" and "D" lacks statistical significance, suggesting a need for further investigation into their interaction. 
	
\end{document}